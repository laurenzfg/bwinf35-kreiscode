\documentclass[a4paper, DIV=12, firstfoot=false, dvipsnames]{scrreprt}
\usepackage{scrlayer-scrpage}

% Umlaute und Font-Encoding
\usepackage[T1]{fontenc}
\usepackage[utf8]{inputenc}
\usepackage[gen]{eurosym}

% Deutsch
\usepackage[ngerman]{babel}

% Das muss fucking nochmal GEIL aussehen!
% \usepackage{microtype} - Erst im Final actimelisieren

% Mathe
\usepackage{mathtools}
\usepackage{amssymb}
\usepackage{ziffer}
% Bilder
\usepackage{graphicx}
\graphicspath{{Grafiken/}}
\usepackage{wrapfig}
\usepackage{tikz}
\usetikzlibrary{calc}

\usepackage{subcaption}

% Coole Links im PDF
\usepackage[hidelinks]{hyperref}

% Aliase zum abdrucken von Shell-CMDs
\newcommand{\shellcmd}[1]{\texttt{\$ #1}\\}
\newcommand{\shellout}[1]{\texttt{#1}\\}

% Alias für Aufgabenname
\newcommand{\task}[1]{Kreis-Code}

% Alias für 2D-Vektoren
\newcommand{\vectwo}[2]{\(\begin{pmatrix}#1\\#2\end{pmatrix}\)}

\ihead{\task} \ohead{Laurenz Grote, Teilnahme 6745 (Team 00001)}
\renewcommand{\chapterpagestyle}{scrheadings}
\pagestyle{scrheadings}

\begin{document}
	\titlehead{Teilnahme 6745 (Team 00001) \hfill Laurenz Grote}
	\title{\task}
	\subtitle{Aufgabe 3}
	\author{Laurenz Friedrich Grote}
	\date{}
	\publishers{		
		\centering
		\includegraphics[width=0.8\linewidth]{frontmatter.png}
	}
	\maketitle
	\tableofcontents
	\vspace {2em}
	Meine Umsetzung für "`\task"' erfolgte unter Ubuntu mit Java (OpenJDK 1.8). Ich habe Ihnen den Programmcode sowie eine ausführbare .jar-Datei beigelegt.
	\pagebreak
	% ----------------------------------------------------------------------------
	\chapter{Erkennung von Grafiken}
		\section{Lösungsidee}
Um möglichst effizient Kreismittelpunkte zu bestimmen, mache ich mir zunächst folgende Eigenschaft von Kreisen zunutze: Am Mittelpunkt eines Kreises ist der Abstand (Radius) zu der äußeren Umrandung des Kreises in jede Richtung gleich.

In einem Bild lassen sich solche Punkte finden, indem man von jedem schwarzen Bildpunkt misst, wie weit man sich auf dem Bild in vertikale wie horizontale Richtung "`bewegen"' kann, ohne auf ein weißes Feld zu stoßen. Wenn dieser Abstand nach rechts, links, oben und unten gleich groß ist, genügt der Punkt der erstgenannten Eigenschaft von Kreisen. 

Von einem Vergleich des Abstandes in weitere Richtungen (wie z.B. in diagonale Richtung) sollte man bei einem Bild aus Pixeln absehen, da bei der Speicherung eines Kreises als Bitmap aus quadratischen Pixeln diagonale Messungen oder gar Messungen unter beliebigem Winkel ungenaue Ergebnisse liefern. Zwar sind solche Messungen prinzipiell möglich, liefern aber anders als rein vertikale oder horizontale Messungen kein ganzzahligen Ergebnisse, da eine Pixeldiagonale \(\sqrt{2}\) Pixelseiten lang ist.
\footnote{: \( \vert \begin{pmatrix}1\\1\end{pmatrix} \vert = \sqrt{2}\)}

\begin{figure}[!ht]
	\centering	
	\includegraphics[width=0.8\textwidth]{durchmesservergleich}
	\caption{Versch. Formen mit eingezeichnetem Mittelpunkt (erstes Kriterium)}
\end{figure}

Da diese Eigenschaft jedoch auch auf Punkte in Quadraten und anderen unregelmäßigen Formen zutrifft (s. Grafik), muss für eine zuverlässige Erkennung eine zweite Eigenschaft von Kreisen genutzt werden: Bei bekanntem Durchmesser kann die Fläche eines Kreises mit der Kreisformel\footnote{: \(\frac{\pi d^2}{4}\)} bestimmt werden. Diese Soll-Fläche kann mit der tatsächlichen Fläche verglichen werden. Sollte das Delta zwischen diesen beiden Flächengrößen nahe 0 sein, handelt es sich mit sehr großer Wahrscheinlichkeit um einen Kreismittelpunkt.

Da die Wahrscheinlichkeit eines False Positives nach Überprüfen beider Kriterien sehr gering ist, nehme ich an, dass jeder Punkt, auf den beide Kriterien zutreffen, ein Kreismittelpunkt ist. Allerdings ist es aufgrund von Kompressionsartefakten o.ä. manchmal möglich, dass es mehrere Punkte gibt, auf die diese Bedingungen zutreffen. Damit der Mittelpunkt eindeutig bestimmt wird, merkt sich der Algorithmus bei der Flächenermittlung die Felder, von denen bereits die Fläche ermittelt wurde. Anschließend kann bei Feldern, deren Fläche schon einmal ermittelt wurde die weitere Überprüfung abgebrochen werden. Schließlich wurde entweder schon ein Mittelpunkt gefunden oder es hat sich herausgestellt, dass es sich nicht um einen Kreis handelt.

\begin{wrapfigure}{r}{0.35\textwidth}
  \centering
  \begin{tikzpicture}[scale=0.3]
\fill[black!90!white] (0,0) circle [radius=1.5];
\fill[black!90!white, even odd rule] (0,0) circle[radius=2.5] circle[radius=3.5];
\fill[black!60!white, even odd rule] (0,0) circle[radius=4.5] circle[radius=5.5];

\foreach \angle in {0, 22.5, 45, 67.590, 90, 112.5, 135, 157.5, 180, 202.5, 225, 247.5, 270, 292.5, 315, 337.5} 
	\draw[thick] (\angle:4.5) -- (\angle:5.5);

\fill[red] (0,0) circle[radius=0.2];

\draw[color=blue!60!violet, thick] (-1.5,1.4) -- ++(0, 0.2) -- ++(1.5, 0) +(0, 0.4)node[black]{\tiny{\(3u\)}}+(0, 0) -- ++(1.5, 0) -- ++(0, -0.2);

\draw[color=blue!60!violet, thick] (0,0) -- (0,-6.7) -- +(-1.5, 0) -- (-1.5,0);
\draw[color=blue!60!violet, thick] (0,0) -- (0,-8) -- +(-2.5, 0) -- (-2.5,0);
\draw[color=blue!60!violet, thick] (0,0) -- (0,-9.3) -- +(-3, 0) -- (-3,0);

\draw (-0.75, -6) node{\tiny{\(\frac{3}{2}u\)}};
\draw (-1.25, -7.4) node{\tiny{\(\frac{5}{2}u\)}};
\draw (-1.25, -8.7) node{\tiny{\(\frac{3}{1}u\)}};
\end{tikzpicture}
  \caption{Größenverhältnisse}
  \label{abb:dims}
\end{wrapfigure}
Danach gilt es noch zu bestimmen, ob es sich bei dem Kreis um einen Mittelpunkt eines \task{}s handelt, schließlich sollen Mittelpunkte sonstiger Kreise im Bild nicht ausgegeben werden. Hierzu können wir uns die Proportionen eines \task{}s zu nutze machen:

Der Durchmesser des Kreises ist als \(3u\) definiert, der Kreisring und die Zwischenräume haben eine Breite von \(1u\). Daraus folgt, dass wenn man den Mittelpunkt in eine beliebige Richtung um \(3u\) verschiebt, der neue Punkt im mittleren, durchgehend schwarzen, Kreisring liegen muss.

Um nun das Vorhandensein eines mittleren Ringes zu überprüfen, nehmen wir vier Punktproben vor. Wir verschieben den Mittelpunkt in alle Himmelsrichtugen um \(u\). Dort vergleichen wir die Länge der entsprechenden linearen Zusammenhangskomponente mit dem Soll \(1u\). Weitere Punktproben sind prinzipiell möglich, jedoch würden diese die Laufzeit erhöhen. \textbf{Ggfs. weitere Flood-Fill auf Fläche des Kreisringes hinzufügen!}

Die Wahrscheinlichkeit eines Erkennungsfehlers ist nach dem Durchmesservergleich, dem Flächenvergleich und den vier Punktproben sehr gering. Im Beispielbild, dass ich um einen Kreis, der kein \task{} ist ergänzt habe, liegt die Erkennungsrate bei 100\%. Abschließend gibt der Algorithmus eine Liste über alle Kreismittelpunkte aus.

\pagebreak
\section{Umsetzung}
%In meiner Implementierung muss das Delta zwischen der Ist- und Soll-Fläche kleiner als 5\% sein.

Zunächst implementierte ich mithilfe von ImageIO eine Bildeinleseprozedur. Da ImageIO nur RGB-JP(E)Gs, PNGs, BMPs und GIFs einlesen kann, habe ich zusätzlich einen Wrapper für ImageMagick\footnote{\url{http://www.imagemagick.org/}, GPLv3-kompatible freie Lizenz. Enthalten in den Repositories der gängigen Linux-Distributionen, Binarys für weitere Betriebssysteme auf der Entwicklerseite.} geschrieben. Wenn dieser über die entsprechende Checkbox in der GUI zugeschaltet wird, können alle gängigen Bildformate gelesen werden. Nachteil ist eine deutlich längere Einlesezeit. Außerdem muss ImageMagick lokal installiert sein und mit dem Befehl \texttt{convert} aufrufbar sein.\footnote{Getestet unter Arch Linux und Ubuntu Linux 16.10. Weitere Betriebssysteme werden vermutlich auch unterstützt.}

Zunächst überlegte ich mir eine möglichst effiziente Datenstruktur für Grafiken, da die Interaktion über ImageIO mit Bitmaps nicht sonderlich effizient ist. Da für die Erkennung von Kreisen in einer Grafik genaue Informationen über die Farbe eines Bildpunktes nicht relevant sind, kann das Bild beim Einlesevorgang in ein boolesches 2D-Array überführt werden: In diesem Array, das die gleiche Größe wie das eingelesene Bild besitzt, sind die Bildpunkte als True gespeichert, die Teil eines \task{}s sein könnten. In der in Teilaufgabe 1 gegebenen Schwarz-Weiß-Grafik ist diese Einstufung noch simpel: Schwarze Bildpunkte können Teil eines \task{}s sein, sonstige nicht.

Um in diesem Array die Punkte zu bestimmen, deren Kreisradien sich in alle vier Richtungen gleichen, bestimme ich zunächst die Länge von aufeinanderfolgenden Streifen aus vorhel als True markierten Feldern. Diese nenne ich nun lineare Zusammenhängigkeitskomponenten. In seperaten Arrays für horizontale und vertikale Streifen speichere ich für jedes Feld die gesamte Länge seiner linearen Zusammenhängigkeitskomponente. \textbf{Zeichnung?} 
Für jedes True-Feld liegt die Länge der Wert in dem Array für horizontale Zusmmenhängikeitskomponenten in \(1 \le l \le width\), für vertikale entsprechend in \(1 \le l \le height\). Alle Feldern, die vorher als False markiert wurden, haben in den Arrays einen Wert von 0. 

Die erste Eigenschaft aus der Aufgabenstellung lässt sich mit der soeben vorberechneten Information nun dahingehend vereinfachen, dass nach aufeinander liegenden Mittelpunkten gleich langer Zusammenhängigkeitskomponenten gesucht wird. Dann ist die erste Eigenschaft gegeben, da in alle Richtungen die Länge gleich ist.

Dafür bestimme ich zunächst alle Mittelpunkte horizontaler linearer Zusammenhängigkeitskomponenten. Dies geschieht, indem ich mit einer linearen Suche über das ganze Bild alle Koordinaten bestimme, deren Zusammenhängigkeitskomponentenlänge größer null ist und der kein Feld vorausgeht, dass Teil einer Zusammenhängigkeitskomponente ist. Feldern am linken Rand gehen grundsätzlich keine Zusammenhängigkeitskomponentenfelder voraus, d. h. dass Komponenten sich meiner Definition nach nicht über Zeilengrenzen hinweg erstrecken. Diese Funktion liefert erst einmal Anfangsstellen von horizontalen Zusammenhängigkeitskomponenten. Wenn ich auf diese Anfangsstelle allerdings nun die Hälfte der Länge der Komponente hinzuaddiere, erhalte ich die x-Koordinate des Mittelpunktes der Zusammenhängigkeitskomponente. Die y-Koordinate ist für eine \textit{horizontale} Komponente durchgehend gleich und daher aus der linearen Suche gegeben.

An der soeben bestimmten Stelle lese ich nun die Länge der vertikalen linearen Zusammenhängigkeitskomponente aus. Wenn die Differenz zwischen dieser und der Länge der horizontalen Komponente 0 oder 1 ist, werte ich die Längen der Komponenten als gleich. Ein Fehler von 1 kann ich durch Rundungsfehler bei ganzzahliger Division entstehen.

Anschließend überprüfe ich noch, ob der Mittelpunktskandidat auch der Mittelpunklt der vertikalen Komponente ist. Dazu verschiebe ich den Mittelpunkt um eine halbe vertikale Komponentenlänge - 1 (halber Durchmesser = Radius) nach obenz bzw. unten. Da dieser Wert kleiner als der Kreisradius ist, muss der Punkt noch innerhalb des Kreises liegen und daher schwarz sein. 

Von diesem Punkt aus kann die Soll-Fläche mit der Kreisformel berechnet werden, denn derer Durchmesser des Kreises ist nun mit der Länge einer der beiden Zusammenhängigkeitskomponenten gegeben. Die tatsächliche Länge der Fläche kann mit einer Flood-Fill ermittelt werden.

Schließlich muss noch das Vorhandensein des Kreisringes wie in der Lösungsidee beschrieben verifiziert werden. Dafür wird an den Punkten die Abweichung vom Soll ermittelt und arithmetisch gemittelt. Auch hier akzeptiert das Programm eine Abweichung im Mittel von 5\%, da durch Anti-Aliasing oder Unschärfen bei der Bildaufnahme Abweichungen vom mathematischen Ideal der Lösungsidee zwangsläufig auftreten.
\pagebreak
\section{Beispiele}
Die erkannten Kreismittelpunkte sind mit violetten Kreuzen markiert.
\begin{figure}[!ht]
	\centering	
	\includegraphics[width=\textwidth]{sek1bsp1}
	\caption{Beispielbild d. Aufgabenstellung}
\end{figure}
	\chapter{Dekodiermodul}
		\section{Lösungsidee}
\begin{wrapfigure}{r}{0.35\textwidth}
  \centering
  \begin{tikzpicture}[scale=0.3]
	\fill[black!90!white] (0,0) circle [radius=1.5];
	\fill[black!90!white, even odd rule] (0,0) circle[radius=2.5] circle[radius=3.5];
	\fill[black!60!white, even odd rule] (0,0) circle[radius=4.5] circle[radius=5.5];

	\foreach \angle in {0, 22.5, 45, 67.590, 90, 112.5, 135, 157.5, 180, 202.5, 225, 247.5, 270, 292.5, 315, 337.5} 
    	\draw[thick] (0,0) -- (\angle:5.5);

	\fill[red] (0,0) circle[radius=0.2];
\end{tikzpicture}

  \caption{Geraden der Kreisringsegmente}
  \label{abb:spidergrafik}
\end{wrapfigure}
Zunächst habe ich den \task{} mit TikZ nachgezeichnet, um den genauen Aufbau durch Nachbau zu verstehen. Die 16 gleichgroßen Kreissegmente sind durch Geraden getrennt, die im Abstand von \(22,5^{\circ}\) vom Pol (dem Mittelpunkt) aus gezeichnet werden.

Für die Dekodierung stehen mir aus dem Kreismittelpunkterkennungsprozess noch folgende Informationen zur Verfügung: Ein Liste der Kreismittelpunkte und der Kreisdurchmesser. Die Kreisdurchmesser entsprechen der Länge der Zusammenhangskomponenten an den Mittelpunktskoordinaten. Aus diesen Informationen kann ich dank bekannter Proportionen eines \task{}s auf die Positionen der Ringe schließen (siehe Abb. \ref{abb:dims}). Wie in der Lösungsidee zur Kreismittelpunkterkennung festgestellt, beträgt gilt \(u=\frac{1}{3}d\).

\begin{figure}[!ht]
	\centering	
	\begin{tikzpicture}[scale=0.5]
	\clip (-1,-0.4) rectangle (6, 3);
	\fill[black!60!white, even odd rule] (0,0) circle[radius=4.5] circle[radius=5.5];

	\foreach \angle in {0, 22.5, 45, 67.590, 90, 112.5, 135, 157.5, 180, 202.5, 225, 247.5, 270, 292.5, 315, 337.5} 
    	\draw[thick] (\angle:4.5) -- (\angle:5.5);

	\filldraw[fill=green!20,draw=green!50!black] (0,0) -- (2,0) arc[start angle=0, end angle=22.5, radius=2];
	\draw[green!50!black, thick] (0,0) -- (22.5:4.5);

	\draw (2, 1.3) node {\(4,5u\)};
	\draw (2.7, 0.4) node {\(22,5^{\circ}\)};

	\draw[very thin] (-6, -6) grid (6, 6);
	\draw[thick] (-6, 0) -- (6, 0);
	\fill[red] (0,0) circle[radius=0.2];
\end{tikzpicture}
	\caption{Punkt einer Gerade, eine Kästchenseite entpricht einem \(u\)}
	\label{abb:trigon}
\end{figure}

Insbesondere kann ich mithilfe von polaren Koordinaten\footnote{\url{https://www.lernhelfer.de/schuelerlexikon/mathematik-abitur/artikel/polarkoordinatensystem}} die Strecken, die die Kreisringe in Segmente einteilen, bestimmen. Jede der Strecken ist ein Teile einer Geraden, die vom Mittelpunkt ausgehend in einem Vielfachen von \(22,5^{\circ}\) durch den Kreis läuft. Die eigentlichen Strecken, die auf dem Kreisring liegen, beginnen nach einem Abstand von \(4,5u\) und enden bei \(5,5u\). Die Punkte aller Strecken lauten also:

\begin{displaymath}
A(k:4,5u) \hspace{2em} B(k:5,5u) \hspace{2em} k := \{22,5 \cdot x \in \mathbb{N}_0 \cup 0 \le x \le 359\}
\end{displaymath}

Nun können wir diese Koordinatenmenge unter Ausnutzung der Regeln im rechtwinkligen Dreieck (s. Abb \ref{abb:trigon}) die kartesischen Koordinaten bestimmen. Das Koordinatensystem ist in der bereits definierten Einheit \(u\).

\begin{gather}
k := \{22,5 \cdot x \in \mathbb{N}_0 \cup 0 \le x \le 359\} \\
\begin{split}
x &= cos(k) \cdot 4,5 \\
y &= sin(k) \cdot 4,5
\end{split}
\hspace{5em}
\begin{split}
x &= cos(k) \cdot 5,5 \\
y &= sin(k) \cdot 5,5
\end{split}
\end{gather}

Da es sehr schwer ist, mit Kreisringsegmenten zu rechnen, nehme ich auch hier eine Vereinfachung vor: Mit den Punkten der segmentbegrenzenden Strecken mit den äußeren Umrandungen des Kreisringes lässt sich 16 Trapeze aufspannen, die den Großteil der Kreissegmente abdecken.:
\begin{figure}[!ht]
	\centering	
	\begin{tikzpicture}[scale=0.5]
	\fill[black!90!white] (0,0) circle [radius=1.5];
	\fill[black!90!white, even odd rule] (0,0) circle[radius=2.5] circle[radius=3.5];
	\fill[black!60!white, even odd rule] (0,0) circle[radius=4.5] circle[radius=5.5];

	\foreach \angle in {0, 22.5, 45, 67.590, 90, 112.5, 135, 157.5, 180, 202.5, 225, 247.5, 270, 292.5, 315, 337.5} 
    	\draw[blue, thick] (\angle:4.5) -- (\angle:5.5) -- (\angle+22.5:5.5) -- (\angle+22.5:4.5) -- cycle;

    	
	\draw[very thin] (-6, -6) grid (6, 6);
	\fill[red] (0,0) circle[radius=0.2];
\end{tikzpicture}
	\caption{Einteilung der 16 Kreisringsegmente in 16 Trapeze}
\end{figure}
\section{Umsetzung}
\section{Beispiele}
	\chapter{Erkennung von Fotos}
		\section{Lösungsidee}
\begin{wrapfigure}{r}{0.45\textwidth}
	\setlength\intextsep{0pt}
	\centering	
	\includegraphics[width=0.4\textwidth]{Grafiken/sek3abb1}
	\caption{Verbesserte Bilderkennung}
	\label{abb:transform}
\end{wrapfigure}
Die in Kapitel 1 vorgestellte Einleseprozedur wird den Anforderungen für die Erkennung eines Fotos oder eines Scans nicht gerecht. Bei schwankender Ausleuchtung des Bildes lässt sich kein geeigneter Schwellwert bestimmen. Stattdessen wende ich verschiedene Algorithmen des maschinellen Sehens an, um ein möglichst ideales Bild aus dem Eingabebild zu extrahieren.

Ich extrahiere zunächst alle Kanten des Bildes. Kanten sind Stellen, an denen sich die Farbwerte eines Pixels schlagartig ändern. Den Kantenextraktionsalgorithmus implementiere ich hierbei so, dass die Kante eine lückenlose Linie darstellt.

Mit diesen Kanten habe ich das Bild segmentiert. Zu schwärzende Bildsegmente sind von einer Kante umrandet. Daher fülle ich jedes Segment komplett schwarz (1), während ich den Hintergrund weiß (0) belasse.

Die Färbung basiert auf die folgende Annahme: Der Hintergrund des Bildes ist monoton gleichfarbig. Alle an den Hintergrund angrenzenden Segmente sind Bestandteile eines Kreisringes. Schließlich fand dort eine schlagartige Farbänderung zum Hintergrund statt. Die Felder, die an diese Segmente angrenzen, sind weiß, da sich die Farbe wiederum schlagartig zum Hintergrund zurück geändert hat. Dieses Muster setze ich fort, bis die Farben aller Segmente bestimmt sind.

Die drei Schritte der Bildprozessierung sind in Abbildung \ref{abb:transform} dargestellt.

Vorteil dieser Vorgehensweise gegenüber einem Schwellwertverfahren ist, dass jeder ähnlich gefärbte Bereich durchgehend gleich gefärbt wird. Bei einem Schwellwertverfahren werden bei einem zu kleinen Schwellwert dunklere Bildbereiche komplett geschwärzt, während bei einem zu hohen Schwellwert einige Kreis-Codes nur lückenhaft erkannt werden.
 
\section{Umsetzung}
\subsection{Graustufenbild}
In einem ersten Schritt ermittle ich aus dem farbigen Bild ein Graustufenbild. Dies erfolgt mit einer gewichteten Mittlung aus den Intensitätswerten der drei Primärfarbkanäle. Laut der Norm CIE 1931\footnote{\url{en.wikipedia.org/wiki/Grayscale}} ist der Grauwert mit folgender Formel zu bestimmen:

\begin{equation}
Y = 0,2126R+0,7152G+0,0722G
\end{equation}

\subsection{Canny-Edge-Detector}
Um in diesem Graustufenbild die Kanten zu bestimmen, nutze ich den 1986 von John Canny vorgestellten Canny Edge Detector. Zu diesem habe ich mich auf einer Online-Veröffentlichung der Universität von Edinburgh informiert: \url{http://homepages.inf.ed.ac.uk/rbf/HIPR2/canny.htm} Dort finden Sie weitere Erläuterungen zu dem Canny-Edge-Detector, den ich im folgenden kurz darstelle.

\subsubsection{Weichzeichnung (Listing \ref{lst:gauss})}
In einem ersten Schritt filtere ich vor der eigentlichen Kantenerkennung grobe Ausreißer aus dem Bild heraus. Hierfür wende ich einen Gaußschen Weichzeichner an. Ein solcher Weichzeichner funktioniert, indem für jedes Pixel ein gewichteter Mittelwert aus seinem eigenen Grauwert und den Grauwerten seiner Umgebung bestimmt wird. Der Gewichtung wird die Gaußsche Normalverteilung zugrunde gelegt. Ich habe mich für ein Sigma von 3 entschieden. So werden grobe Ausreißer entfernt, der Kantenverlauf bleibt jedoch erhalten. Aus dieser Kurve lässt sich folgende Matrix extrahieren\footnote{\url{http://dev.theomader.com/gaussian-kernel-calculator/}}:
\begin{equation}
	\begin{bmatrix}
	0,031827&0,037541&0,039665&0,037541&0,031827 \\
	0,037541&0,044281&0,046787&0,044281&0,037541 \\
	0,039665&0,046787&0,049434&0,046787&0,039665 \\
	0,037541&0,044281&0,046787&0,044281&0,037541 \\
	0,031827&0,037541&0,039665&0,037541&0,031827 \\
	\end{bmatrix}
\end{equation}
Jeder Pixel wird mit dem mittleren Wert multipliziert. Die umliegenden Pixel werden mit ihren Pendants in der Matrix multipliziert. Die Summe aus allen Produkten entspricht dem neuen Wert des Pixels.

Allerdings kann die Gaußsche Weichzeichnung auch in einen horizontalen und vertikalen Bestandteil aufgeteilt werden. Nach dieser Aufteilung erhält man folgende Matrix:
\begin{equation}
	\begin{bmatrix}
	0,1784&0,210431&0,222338&0,210431&0,1784
	\end{bmatrix}
\end{equation}
Man kann mit dieser Matrix das gleiche Ergebnis erzielen, indem man sie zunächst in horizontale und anschließend in vertikale Richtung anwendet. Diese Vorgehensweise hat eine bessere Laufzeit, da für die Glättung eines Pixels nicht \(4^2\), sondern nur \(2\times 4\) Pixel betrachtet werden müssen.

\subsubsection{Sobel-Operator (Listing \ref{lst:sobel})}
Anschließend wende ich auf das nun geglättete Bild den Sobel-Operator an. Dieser Operator entspricht der 1. Ableitung über die Helligkeitskurve des Bildes. Der Operator wird mithilfe von \textit{Convolution} über das gesamte Bild berechnet. Convolution ist das bereits von dem Gauß-Weichzeichner bekannte Prinzip, das auf jedes Pixel eine Matrix angewandt wird. Der Sobel-Operator basiert auf zwei Convulution-Durchläufen mit folgenden Matrizen:
\begin{gather}
	\begin{split}
		\begin{bmatrix}
			-1,0&-2,0&-1,0\\
			0,0&0,0&0,0\\
			1,0&2,0&1,0\\
		\end{bmatrix}
	\end{split}
	\hspace{5em}
	\begin{split}
		\begin{bmatrix}
			-1,0&0,0&1,0\\
			-2,0&0,0&2,0\\
			-1,0&0,0&1,0\\
		\end{bmatrix}
	\end{split}
\end{gather}

Diese Matrizen entsprechen der Ableitung in vertikale und horizontale Richtung, da die jeweils an das Feld in horizontale oder vertikale Richtungen angrenzenden Felder voneinander subtrahiert werden.
Wenn die umliegenden Pixel den gleichen Intensitätswert haben, ist das Ergebnis des Sobel-Operators 0.
Bei Intensitätsunterschieden verändert sich das Ergebnis des Operators entsprechend.
Da allerdings für die weiteren Berechnungen eine kombinierte Ableitung benötigt wird, müssen beide Ableitungswerte eines Pixels kombiniert werden.

Hierfür können die beiden Ableitungswerte als ein rechtwinkliges Dreieck aufgefasst werden. Die Ausschläge der Ableitungen in x- und y-Richtung entsprechen den beiden Katheten. Ein kombinierter Wert aus beiden Ableitungen entspricht dann der Länge der Hypotenuse. Diese lässt sich mit dem Satz des Pythagoras berechnen (\(G_{xy} = \sqrt{G_x^2 + G_y^2}\)).

\subsubsection{Non-Maximum-Supression (Nichtmaximumsunterdrückung) (Ebenfalls Listing \ref{lst:sobel})}
Leider liefert der Sobel-Operator Kanten, die mehrere Pixel breit sind. Schließlich verlaufen die Kanten in einem Foto nicht vollkommen abrupt, sondern verlaufen über mehrere Pixel. 
Ein Lösungsansatz zur Berechnung von möglichst dünnen Kanten ist die Nichtmaximumsunterdrückung (NMS). 

Da die beiden Ableitungsfunktionen ein rechtwinkliges Dreieck bilden, kann mithilfe der Tangensfunktion der Winkel der Kante ermittelt werden:
\begin{equation}
	tan(\alpha) = \frac{G_y}{G_x}
\end{equation}

Mit diesem Winkel können die Pixel bestimmt werden, die an der gleichen Kante liegen. Wenn der Ableitungswert des Pixels nicht das Maximum seiner Nachbarn darstellt, kann sein Ableitungswert auf 0 gesetzt werden (Der Pixel wird in der Ausgabegrafik unterdrückt). Schließlich gibt es entlang der Kante einen stärken Farbintensitätsunterschied.  

Wenn der Winkel beispielsweise \(0^{\circ}\) beträgt, verläuft die Kante in horizontale Richtung. Dann wird der Ableitungswert des Pixels mit seinem nördlichen und südlichen Nachbarn verglichen.
Nur wenn der Ableitungswert das Maximum von diesen Pixeln darstellt, ist er Teil der 1px breiten Kante. Sonst ändert sich bei diesem Pixel zwar die Farbe. Aber es folgt unmittelbar ein noch stärkerer Farbumschlag.

\subsubsection{Binärisierung mit Hysterese (Listing \ref{lst:hyst})}
Zur Binärisierung des Ergebnisses der Sobel-Operators wende ich eine Technik namens \textit{Hysterese} an. Bei einer Hysterese wird zunächst mit einem hohen Schwellwert das Bild binärisiert. 
Im Kontext meiner Implementierung bedeutet dies, dass alle Pixel mit einem Ableitungswert höher als 40 im Ausgabebild der Canny-Eckenerkennung schwarz gefärbt werden.

Da aber möglicherweise eine Kante auch aus weniger stark abgesetzten Pixeln besteht, akzeptiert die Hysterese für an bereits erkannte Kantenpunkte anliegende Punkte einen niedrigeren Schwellwert. Sobald ein Punkt einer Ecke gefunden wurde, wird diese "`verfolgt"'.

Diese Verfolgung ist mit einem Stack implementiert. Jeder im Erkennungsschritt mit hohem Schwellwert erkannte Pixel wird auf diesen Stack gelegt. Nach Abschluss des Erkennungsschrittes werden alle Pixel, die an ein Pixel aus dem Stack angrenzen, noch nicht gefunden wurden und über dem verringerten Schwellwert von 1 liegen, schwarz markiert. Diese Pixel werden wiederum auf den Stack gelegt, sodass die Kante mit dem verringerten Schwellwert verfolgt wird. 

\subsubsection{Dilation (Listing \ref{lst:dilation})}
Aufgrund der Non-Maximum-Supression sind kleine Lücken in der Grafik entstanden. Dies verhindert eine sinnvolle Ausführung von den in meinem Algorithmus häufig verwendeten Flood-Fills. Daher wende ich auf das Ergebnis der Hysterese Dilation an. Hierbei wird jedes Pixel, das mehr als einen schwarzen Nachbarn hat, geschwärzt.
Damit verdicke ich die Linie. Allerdings ist sie weiterhin weitaus exakter, als sie es ohne Non-Maximum-Supression wäre.

\subsection{Einfärben des Bildes (Listing \ref{lst:ausf})}
Der Canny-Detektor gibt ein Bild aus Kanten aus (Mittlere Grafik in Abb. \ref{abb:transform}). Die weiteren Berechnungsschritte benötigen jedoch ein Bild, in dem der Kreis, der Kreisring und die Segmente komplett schwarz eingefärbt sind. Unter Ausnutzung der Annahme aus der Lösungsidee habe ich einen Algorithmus formuliert.
Dieser nimmt als Eingabe das Ergebnis des Canny-Eckenerkennungsprozesses und hat als Ausgabe ein Binärbild als boolesches 2D-Array. Damit ist das Ausgabeformat der neuen Bildeinleseprozedur identisch zu dem der simplen Einleseprozedur aus Kapitel 1.

Von Pixel(0|0) ausgehend werden alle im Canny-Bild erreichbaren weißen Pixel mithilfe einer Flood-Fill im Ausgabebild als weiß abgespeichert, da diese den Hintergrund darstellen.

Anschließend werden alle Kantenpixel, die an den Hintergrund angrenzen, schwarz markiert. Diese Pixel stellen eine Kante zum Vordergrundbereich dar.
Da sie die Kante zum Vordergrundbereich sind, werden alle an diese Pixel angrenzenden Nicht-Kanten-Pixel, die im Ausgabebild noch nicht markiert wurden, schwarz markiert. Schließlich befindet sich dieser Bereich weiterhin im Vordergrund.
Diese Prozedur wird abwechselnd zur Bestimmung von Vorder- und Hintergrundbereichen eingesetzt.

Praktisch umgesetzt habe ich dies mit einer Entlehnung aus der Graphentheorie.\footnote{\url{https://en.wikipedia.org/wiki/Connected-component_labeling}}. Mein Algorithmus basiert auf der im Abschnitt "`One component at a time"' vorgestellten Idee.

In einem Integer-Array gebe ich jedem Hintergrundpixel den Wert 0. Darauf gebe ich den anliegenden True-Pixeln den Wert 1. Die an diese Kante anliegenden False-Pixel erhalten ebenfalls den Wert 1, da sie wie oben geschildert zu der Kante gehören.

Danach fahre ich mit der nächsten Kantengruppe fort, nur vergebe ich dort den Wert 2. Schlussendlich müssen dann Pixel mit einem ungeraden Wert schwarz gefärbt werden, während Pixel mit einem geraden Wert weiß belassen werden.

\subsection {Rotation der Kreis-Codes (Listing \ref{lst:decode})}
In einem Foto ist nicht davon auszugehen, dass der Nutzer das erste Segment exakt an der x-Achse ausrichtet. Daher muss der Trapezkranz aus Kapitel 2 so über den Kreisring gelegt werden, dass jedes Trapez möglich exakt über einem Kreisringsegment liegt. 

Dies gelingt, indem die Achse so verschoben wird, dass sie auf einer Kante zwischen zwei Kreisringsegmenten liegt. Dafür muss eine beliebige Kante auf dem äußeren Kreisring gefunden werden. An einer solchen Stelle muss in einem Abstand von \(4,7u\) vom Pol (vgl. Abb. \ref{abb:dims}) eine Kante vorliegen. Also muss an dieser Stelle im Ausgabebild von Canny True stehen. Zu beachten ist, dass hier nicht das ausgefüllte Bild genommen werden darf, da nach Kanten, nicht nach Segmenten, gesucht wird.
So suche ich in jedem Winkel vom Pol aus eine Kante.

Dann addiere ich den Winkel \(0^{\circ} \le \delta \le 360^{\circ}\) zwischen der x-Achse und dem Strahl zu einer beliebigen Kante auf jeden der einteilenden Strahlwinkel hinauf. Somit verschiebe ich den gesamten Trapezkranz so, dass er exakt auf den Trapezen liegt. Statt den Formeln \ref{eq:nKoords} nutze ich folgende Formeln:

\begin{gather}
	\begin{split}
		x_1 &= cos(n \cdot \frac{22,5\pi}{180} + \delta) \cdot 5,5u + x_0\\
		y_1 &= sin(n \cdot \frac{22,5\pi}{180} + \delta) \cdot 5,5u + y_0\\ \vspace{2em}
		x_2 &= cos(n \cdot \frac{22,5\pi}{180} + \delta) \cdot 4,5u + x_0\\
		y_2 &= sin(n \cdot \frac{22,5\pi}{180} + \delta) \cdot 4,5u + y_0
	\end{split}
	\hspace{1.2em}
	\begin{split}
		x_3 &= cos(((n+1)\bmod{}16) \cdot \frac{22,5\pi}{180} + \delta) \cdot 5,5u + x_0\\
		y_3 &= sin(((n+1)\bmod{}16) \cdot \frac{22,5\pi}{180} + \delta) \cdot 5,5u + y_0 \\ \vspace{2em}
		x_4 &= cos(((n+1)\bmod{}16) \cdot \frac{22,5\pi}{180} + \delta) \cdot 4,5u + x_0\\
		y_4 &= sin(((n+1)\bmod{}16) \cdot \frac{22,5\pi}{180} + \delta) \cdot 4,5u + y_0
	\end{split} \label{eq:nKoordsNeu}
\end{gather}

Dass dadurch mehr als \(360^{\circ}\) in den Sinus eingegeben werden ist unwesentlich, da sich die Sinuskurve nach \(360^{\circ}\) wiederholt.
\pagebreak
\section{Beispiele}
Allen Optimierungen zum Trotz erkennt mein Programm in den weiteren Beispielen nicht alle Kreis-Codes. Ausgabebilder aller BwInf-Beispiele finden Sie in der Einsendung. Hier gebe ich tabellarisch wieder, wie vollständig das Programm die Beispieleingaben dekodiert hat. \\ 
(\checkmark{} Vollständig, \(\varnothing\) unvollständig, \(\times\) ohne Ergebnis)

\begin{table}[!h]
    \begin{tabular}{lllllllllll}
    Cam 1             & Cam 2           & Cam 3           & Cam 4             & Cam 5           & Cam 6           & Cam 7           & Cam 8             & Cam 9           & Cam A           & Cam B      \\ \hline
    \(\varnothing\) & \checkmark & \checkmark & \(\varnothing\) & \checkmark & \checkmark & \checkmark & \(\varnothing\) & \checkmark & \checkmark & \(\varnothing\)\\
    \end{tabular} \\ \\
    \begin{tabular}{lllllll}
    Bitmap & Noise 25      & Noise 50      & ROT         & CMYK        & Grey (GIF)  & Grey (JPG)  \\ \hline
    \checkmark & \checkmark & \(\times\) & \checkmark & \checkmark & \checkmark & \checkmark \\
    \end{tabular}
    \caption {BwInf-Beispieleingaben}
\end{table}

\subsection{Fehleranalyse}
Auch wenn es mir nicht gelungen ist, das Programm so zu optimieren, dass alle Codes dekodiert werden, habe ich in allen unvollständigen Dekodierungen die Fehlerursache gesucht:

\subsubsection{Noise 50}
\begin{minipage}{0.7\textwidth}
\includegraphics[width=0.9\textwidth]{Grafiken/Ausgaben/noise50}
\end{minipage}
\begin{minipage}{0.3\textwidth}
Dieses Bild ist extrem verrauscht. Nichtsdestotrotz ist eine Erkennung theoretisch möglich, wenn man die Treshold im Quellcode manuell auf 75 anhebt. Leider werden in diesem Fall in zahlreichen anderen Bildern die Kreis-Codes nicht komplett erkannt.
\end{minipage}

\subsubsection{Beispiele 1 und 4}
\begin{minipage}{0.7\textwidth}
\includegraphics[width=0.9\textwidth]{Grafiken/Ausgaben/cam1}
\includegraphics[width=0.9\textwidth]{Grafiken/Ausgaben/cam4}
\end{minipage}
\begin{minipage}{0.3\textwidth}
Hier scheitert mein Algorithmus an den zahlreichen weißen Punkten in den nicht erkannten Kreis-Codes. Eine stärkere Weichzeichnung ist nicht zielführend, da dann die Kantengrenzen verwischen. Ein geringerer Schwellwert sorgt für False Positives an den Farbwechseln.
\end{minipage}

\subsubsection{Beispiel 8}
\begin{minipage}{0.7\textwidth}
\includegraphics[width=0.9\textwidth]{Grafiken/Ausgaben/cam8}
\end{minipage}
\begin{minipage}{0.3\textwidth}
In diesem Bild sind die Kreise so unscharf, dass eine Erkennung der Kanten bei einigen Kreisringen fehlschlägt
\end{minipage}

\subsubsection{Beispiel B}
\begin{minipage}{0.7\textwidth}
\includegraphics[width=0.9\textwidth]{Grafiken/Ausgaben/camB}
\end{minipage}
\begin{minipage}{0.3\textwidth}
In der oberen rechten Hälfte ist das Blatt geknickt worden. Da in meinem Algorithmus keinerlei Maßnahmen zur Begradigung eingebaut sind, kann der Kreis-Code nicht entziffert werden, da die Trapeze nicht auf den Kreisringsegmenten liegen. Auch nach einer Internetrecherche habe ich keinen Algorithmus gefunden, der dieses Problem ohne eine gerade Referenz behebt.
\end{minipage}

\pagebreak
\section{Eigene Beispiele}
In meinen Beispielen habe ich typische Use-Cases von Barcodes evaluiert. Zum einen habe ich Kreis-Codes in ein Textdokument (blatt.png) eingebettet, da insbesondere QR-Codes oft genutzt werden, um z.B. Links anzugeben.
Auch habe ich einen Kreiscode an meine Zimmerwand geheftet. Damit zeige ich, dass die Bildumrandung nicht zwingend Weiß sein muss, sondern auch sonstige ruhige Flächen wie Raufasertapeten ein geeignete Träger für einen Kreis-Code sind.
Zuguterletzt habe ich ein Bild mit roten Kreis-Codes erstellt. Oftmals möchten Nutzer Barcodes kreativ personalisieren, damit sie besser in ihre Umgebung passen.

\begin{center}
\includegraphics[width=0.4\textwidth]{Grafiken/Ausgaben/blatt}
\hspace{1em}
\includegraphics[width=0.4\textwidth]{Grafiken/Ausgaben/farbig}
\end{center}
\begin{center}
\includegraphics[width=0.8\textwidth]{Grafiken/Ausgaben/raufaser}
\end{center}

\section {Operationsbereich des Programmes}
Mit den Erfahrungen aus den BwInf-Beispielen und eigenen Testfällen habe ich untersucht, welche Voraussetzungen gegeben sein müssen, damit die Dekodierung erfolgreich ist:
\begin{itemize}
	\item Der Hintergrund muss eine ruhige, zusammenhängende Fläche sein. Sonst gelingt die Flood-Fill zur Hintergrundmarkierung nicht. Wenn beispielsweise das Blatt Papier auf eine schwarze Unterlage gelegt wird, ist das gesamte Blattinnere invertiert, da die Außenkante des Blattes als Kante eines Kreis-Codes aufgefasst wird. Ruhig bedeutet in diesem Zusammenhang, dass Farben sanft ineinander verlaufen müssen.
	\item Der Sobel-Operator muss an mindestens einer Stelle jeder Linie des Kreis-Codes einen Wert höher als 40 liefern. Sonst wird diese Kante nicht verfolgt. Die Farbe des Kreis-Codes ist beliebig, es muss nur der Kontrast zum Hintergrund hinreichend groß sein.
	\item Das Blatt darf nicht zerknickt oder zerknüllt sein, die Kamera muss senkrecht auf den Kreis-Code zeigen. Schließlich basieren die Formeln für den Trapezkranz auf der Annahme, das das Bild glatt vorliegt. 
	\item Das Rauschen des Kamerabildes darf nicht zu groß sein. Ein Rauschen wie in Bild Noise-50 entsteht allerdings nur in stark verstrahlten Gebieten oder GIMP, daher ist die Entrauschung meines Programms für Alltagssituationen ausreichend.
\end{itemize}
\end{document}
